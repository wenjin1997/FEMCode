\documentclass{ctexart}
% \usepackage[UTF-8]{ctex}
\usepackage{amsmath}   
\usepackage{booktabs}
\usepackage{multirow}
\usepackage{tabularx}

\usepackage{booktabs,multirow,longtable}

\usepackage{listings}
\usepackage{color}

% 插入Matlab代码设置
\usepackage[framed, numbered, autolinebreaks, useliterate]{mcode}

%导言区插入下面三行
\usepackage{graphicx} %插入图片的宏包
\usepackage{float} %设置图片浮动位置的宏包
\usepackage{subfigure} %插 入多图时用子图显示的宏包

% % 参考文献
% % 注意参考文献请用Biber编译,不能使用BiberTex或者BiberTex-8
\usepackage[backend=biber,style=gb7714-2015]{biblatex}
\addbibresource[location=local]{bib/sample.bib}

%% 引用网页
\usepackage{url}


% \definecolor{dkgreen}{rgb}{0,0.6,0}
% \definecolor{gray}{rgb}{0.5,0.5,0.5}
% \definecolor{mauve}{rgb}{0.58,0,0.82}

% \lstset{frame=tb,
%   language=Matlab,
%   aboveskip=3mm,
%   belowskip=3mm,
%   showstringspaces=false,
%   columns=flexible,
%   basicstyle={\small\ttfamily},
%   numbers=left,
%   numberstyle=\tiny\color{gray},
%   keywordstyle=\color{blue},
%   commentstyle=\color{dkgreen},
%   stringstyle=\color{mauve},
%   breaklines=true,
%   breakatwhitespace=true,
%   tabsize=3
% }



\title{CR-P0方法求解二维Navier-Stokes方程}
\author{谢文进}
\date{\today}
\begin{document}
\maketitle
\section{理论推导}
\subsection{问题描述}
对于Navier-Stokes问题

\begin{equation}
    \label{NavierStokes}
    \left\{\begin{matrix}
        \partial_t\mathbf{u} + (\mathbf{u} \cdot \nabla)\mathbf{u}    - 
\nu \Delta \mathbf{u}  + \nabla p = f, \quad \text{in } \Omega\\ 
        \nabla \cdot \mathbf{u} = 0 = h, \quad \text{in } \Omega \\
        \mathbf{u} = g = 0, \quad \text{on } \partial \Omega
\end{matrix}\right.
\end{equation}
其中,取$\nu = 1$。这里函数$h(\mathbf{x},t)$为预留函数接口,方便后续处理随机项。

第一步,先求解如下方程
\begin{equation}
    \label{NavierStokes01}
    \left\{\begin{matrix}
        (\mathbf{u} \cdot \nabla)\mathbf{u}    - 
 \nu \Delta \mathbf{u}  + \nabla p = f, \quad \text{in } \Omega\\ 
         \nabla \cdot \mathbf{u} = 0 = h, \quad \text{in } \Omega \\
         \mathbf{u} = g = 0, \quad \text{on } \partial \Omega
 \end{matrix}\right.
\end{equation}
其中
$$
\nabla = \begin{pmatrix}
    \partial _x\\
   \partial _y
   \end{pmatrix},
$$

$$
(\mathbf{u} \cdot \nabla)\mathbf{u} = \begin{pmatrix}
    u_1 \partial _x + u_2 \partial _y
    \end{pmatrix}
    \begin{pmatrix}
     u_1\\
    u_2
    \end{pmatrix}
    =\begin{pmatrix}
     u_1\partial _xu_1+u_2\partial _yu_1\\
     u_1\partial _xu_2+u_2\partial _yu_2
    \end{pmatrix}
$$

对于此非线性问题,需要使用牛顿迭代方法,即找到$u^{(l)} \in H^1(\Omega) \times H^1(\Omega)$,
$p^{(l)} \in L^2(\Omega)$使得
\begin{equation}
    \left\{\begin{matrix}
        c(\mathbf{u}^{(l)}, \mathbf{u}^{(l-1)}, \mathbf{v} ) 
+ c(\mathbf{u}^{(l-1)}, \mathbf{u}^{(l)}, \mathbf{v} ) +
a(\mathbf{u}^{(l)} , \mathbf{v}) 
+  b(\mathbf{v},p^{(l)} )\\
= (f,\mathbf{v}) + c(\mathbf{u}^{(l-1)}, \mathbf{u}^{(l-1)}, \mathbf{v} ),\\ 
        b(\mathbf{u}^{(l)}, q ) = 0 = (h, q) .
\end{matrix}\right.
\end{equation}
其中
$$
a(\mathbf{u,v})=\nu(\nabla \mathbf{u},\nabla \mathbf{v}), 
\quad b(\mathbf{v},p)=-(\text{div } \mathbf{v},p)
$$
和用CRP0方法求解Stokes方程中的算子一样,而
$$
c(\mathbf{w,u,v})= ((\mathbf{w} \cdot \nabla)\mathbf{u}, \mathbf{v}).
$$

\subsection{常用等式}
常用向量展开:
$$
\vec{u}_h = \begin{pmatrix}
    \phi_1&  \phi_2& \phi_3 & 0 & 0 & 0\\
    0 & 0 & 0 & \phi_1 & \phi_2 & \phi_3
  \end{pmatrix} 
  \begin{pmatrix}
   u_{11}^0\\
   u_{12}^0\\
   u_{13}^0\\
   u_{21}^0\\
   u_{22}^0\\
   u_{23}^0
  \end{pmatrix},
$$

$$
\nabla u_i = \begin{pmatrix}
    \nabla \phi_1& \nabla \phi_2 & \nabla \phi_3
  \end{pmatrix}_{2 \times 3}
  \begin{pmatrix}
   u_{i1}\\
   u_{i2}\\
  u_{i3}
  \end{pmatrix},i=1,2
$$

$$
v_i = 
\begin{pmatrix}
  v_{i1}& v_{i2}  & v_{i3}
\end{pmatrix}
\begin{pmatrix}
 \phi _1\\
  \phi _2\\
 \phi _3
\end{pmatrix}
\quad i=1,2.
$$

常用重心坐标函数积分:
\begin{equation}
    \int _K \boldsymbol{\lambda} ^\alpha (\boldsymbol{x})d\boldsymbol{x} = 
    \frac{\alpha!n!}{(|\alpha|+ n)!} |K|.
\end{equation}
其中$\alpha$是多重指标,$\alpha = (\alpha_1,\alpha_2,\cdots,\alpha_k)$。$\alpha$的长度$|\alpha|=\sum_{i=1}^{k}\alpha _i $,
$\alpha! = \alpha_1!\alpha_2!\cdots \alpha_k!$。对于一个给定的向量$\boldsymbol{x}=(x_1,x_2,\dots,x_k)$,$\boldsymbol{x}^\alpha
= x_1^{\alpha_1}x_2^{\alpha_2}\cdots x_k^{\alpha_k}$。最后设$\boldsymbol{\lambda}=(\lambda_1,\lambda_2,\cdots,\lambda_{n+1})$
表示重心坐标向量,$n$表示空间的维数。

对于维数为2的情况,有
\begin{align*}
    \iint_K \lambda _i^2dxdy & = \frac{2!2!}{(2+2)!} |K| \\
    & = \frac{4}{4!}|K|\\
    & = \frac{1}{6}|K|  
\end{align*}

\begin{align*}
    \iint_K \lambda _idxdy & = \frac{2!}{(1+2)!} |K| \\
    & = \frac{2}{3!}|K|\\
    & = \frac{1}{3}|K|  
\end{align*}

当$i \neq j$时
\begin{align*}
    \iint_K \lambda _i \lambda _jdxdy  & = \frac{2!}{(2+2)!} |K| \\
    & = \frac{2}{4!}|K|\\
    & = \frac{1}{12}|K|
\end{align*}

下面推导在三角形单元上基函数积分$\iint_K \phi_i\phi_jdxdy$,由上述重心坐标函数积分得

当$i=j$时
\begin{align*}
    \iint_K \phi _i \phi_jdxdy  & = \iint_K \phi _i^2dxdy \\
    & = 4\iint_K(\lambda _i - \frac{1}{2})^2dxdy\\
    & = 4\iint_K(\lambda _i^2 -\lambda _i+ \frac{1}{4} )dxdy\\
    & = 4|K|(\frac{1}{6}-\frac{1}{3} + \frac{1}{4})\\
    & = 4|K|(\frac{2}{12}-\frac{4}{12} + \frac{3}{12})\\
    & = \frac{|K|}{3},
\end{align*}

当$i \neq j$时
\begin{align*}
    \iint_K \phi _i \phi_jdxdy  & = 4\iint_K(\lambda _i - \frac{1}{2})(\lambda _j - \frac{1}{2})dxdy \\
    & = 4\iint_K(\lambda _i\lambda _j - \frac{1}{2}\lambda _i - \frac{1}{2}\lambda _j + \frac{1}{4})dxdy\\
    & = 4|K|(\frac{1}{12}-\frac{1}{2} \times \frac{1}{3} -\frac{1}{2} \times \frac{1}{3} + \frac{1}{4})\\
    & = 4|K|(\frac{1}{12}-\frac{2}{12} -\frac{2}{12} + \frac{3}{12})\\
    & = 0.
\end{align*}



\subsection{推导$c(\mathbf{u}_h^{(l)}, \mathbf{u}_h^{(l-1)}, \mathbf{v}_h)$,矩阵NA}
为了方便描述,这里假设牛顿迭代前一步得到的为$\mathbf{u}_h^0$,现在要求的是$\mathbf{u}_h$,则

\begin{align*}
    & c(\mathbf{u}_h^{l}, \mathbf{u}_h^{(l-1)}, \mathbf{v}_h)\\
    & = \iint_K(\vec{u} \cdot \nabla) \vec{u}^0vdxdy \\
    & = \iint_K(u_1 \partial x + u_2 \partial _y)\begin{pmatrix}
     u_1^0\\
     u_2^0
    \end{pmatrix} \cdot \begin{pmatrix}
     v_1\\
    v_2
    \end{pmatrix}dxdy\\
    & = \iint_K\begin{pmatrix}
     u_1 \partial xu_1^0 + u_2 \partial _y u_1^0\\
     u_1 \partial xu_2^0 + u_2 \partial _y u_2^0
    \end{pmatrix} \cdot \begin{pmatrix}
     v_1\\
    v_2
    \end{pmatrix}dxdy\\
    & = \iint_K(\vec{u} \cdot (\nabla u_1^0) )v_1+(\vec{u} \cdot (\nabla u_2^0))v_2dxdy
\end{align*}

展开得到

\begin{align*}
    & c(\mathbf{u}_h^{l}, \mathbf{u}_h^{(l-1)}, \mathbf{v}_h)\\
    & = \iint_K
\begin{pmatrix}
    \phi_1&  \phi_2& \phi_3 & 0 & 0 & 0\\
    0 & 0 & 0 & \phi_1 & \phi_2 & \phi_3
  \end{pmatrix} 
  \begin{pmatrix}
   u_{11}\\
   u_{12}\\
   u_{13}\\
   u_{21}\\
   u_{22}\\
   u_{23}
  \end{pmatrix} \\
  & \cdot
(\nabla \phi_1 u_{11}^0 + \nabla \phi_1 u_{12}^0 + \nabla \phi_1 u_{13}^0 )\begin{pmatrix}
  v_{11}& v_{12} & v_{13}
\end{pmatrix}
\begin{pmatrix}
 \phi_1\\
 \phi_2\\
 \phi_3
\end{pmatrix}\\
& +
\begin{pmatrix}
    \phi_1&  \phi_2& \phi_3 & 0 & 0 & 0\\
    0 & 0 & 0 & \phi_1 & \phi_2 & \phi_3
  \end{pmatrix} 
  \begin{pmatrix}
   u_{11}\\
   u_{12}\\
   u_{13}\\
   u_{21}\\
   u_{22}\\
   u_{23}
  \end{pmatrix} \\
  & \cdot
(\nabla \phi_1 u_{21}^0 + \nabla \phi_1 u_{22}^0 + \nabla \phi_1 u_{23}^0 )\begin{pmatrix}
  v_{21}& v_{22} & v_{23}
\end{pmatrix}
\begin{pmatrix}
 \phi_1\\
 \phi_2\\
 \phi_3
\end{pmatrix}dxdy\\
&=\begin{pmatrix}
  v_1^T&v_2^T
\end{pmatrix}
\iint_K \\
& \begin{bmatrix}
 \begin{pmatrix}
 \phi_1\\
 \phi_2\\
 \phi_3
\end{pmatrix}
\begin{pmatrix}
 u_{11}^0 &  u_{12}^0  & u_{13}^0 
\end{pmatrix}
\begin{pmatrix}
 \nabla \phi_1\\
 \nabla \phi_2\\
 \nabla \phi_3
\end{pmatrix}
\begin{pmatrix}
  \phi_1& \phi_2 & \phi_3 & 0 & 0 & 0\\
  0&  0&  0& \phi_1 & \phi_2 & \phi_3
\end{pmatrix}
\\
\begin{pmatrix}
 \phi_1\\
 \phi_2\\
 \phi_3
\end{pmatrix}
\begin{pmatrix}
 u_{21}^0 &  u_{22}^0  & u_{23}^0 
\end{pmatrix}
\begin{pmatrix}
 \nabla \phi_1\\
 \nabla \phi_2\\
 \nabla \phi_3
\end{pmatrix}
\begin{pmatrix}
  \phi_1& \phi_2 & \phi_3 & 0 & 0 & 0\\
  0&  0&  0& \phi_1 & \phi_2 & \phi_3
\end{pmatrix}
\end{bmatrix}dxdy\vec{u} 
\end{align*}

记
$$
sumUvec1 = \begin{pmatrix}
 u_{11}^0 &  u_{12}^0  & u_{13}^0 
\end{pmatrix}
\begin{pmatrix}
 \nabla \phi_1\\
 \nabla \phi_2\\
 \nabla \phi_3
\end{pmatrix}
$$

$$
sumUvec2 = \begin{pmatrix}
    u_{21}^0 &  u_{22}^0  & u_{23}^0 
   \end{pmatrix}
   \begin{pmatrix}
    \nabla \phi_1\\
    \nabla \phi_2\\
    \nabla \phi_3
   \end{pmatrix}
$$

我们可以知道矩阵$sumUvec1$及$sumUvec2$均为$1\times 2$矩阵,记$sumUvec1(i)$及$sumUvec2(i)$为矩阵的第一行第$i$列元素。
下面计算推算矩阵NA:

\begin{align*}
    & NA\\
    & = \iint_K \\
    &
    \begin{bmatrix}
    \begin{pmatrix}
     \phi_1\\
     \phi_2\\
     \phi_3
    \end{pmatrix}
    \begin{pmatrix}
      sumUvec1(1)&sumUvec1(2)
    \end{pmatrix}
    \begin{pmatrix}
      \phi_1& \phi_2 & \phi_3 & 0 & 0 & 0\\
      0&  0& 0 & \phi_1 & \phi_2 & \phi_3
    \end{pmatrix}
     \\
    \begin{pmatrix}
     \phi_1\\
     \phi_2\\
     \phi_3
    \end{pmatrix}
    \begin{pmatrix}
      sumUvec2(1)&sumUvec2(2)
    \end{pmatrix}
    \begin{pmatrix}
      \phi_1& \phi_2 & \phi_3 & 0 & 0 & 0\\
      0&  0& 0 & \phi_1 & \phi_2 & \phi_3
    \end{pmatrix} 
    \end{bmatrix}\\
    & dxdy \\
    & = \iint_K \\
    &
    \begin{bmatrix}
    \begin{pmatrix}
      \phi_1 sumUvec1(1)&  \phi_1 sumUvec1(2)\\
      \phi_2 sumUvec1(1)&  \phi_2 sumUvec1(2) \\
      \phi_3 sumUvec1(1)&  \phi_3 sumUvec1(2)
    \end{pmatrix}
    \begin{pmatrix}
      \phi_1& \phi_2 & \phi_3 & 0 & 0 & 0\\
      0&  0& 0 & \phi_1 & \phi_2 & \phi_3
    \end{pmatrix}   
    \\
    \begin{pmatrix}
      \phi_1 sumUvec2(1)&  \phi_1 sumUvec2(2)\\
      \phi_2 sumUvec2(1)&  \phi_2 sumUvec2(2) \\
      \phi_3 sumUvec2(1)&  \phi_3 sumUvec2(2)
    \end{pmatrix}
    \begin{pmatrix}
      \phi_1& \phi_2 & \phi_3 & 0 & 0 & 0\\
      0&  0& 0 & \phi_1 & \phi_2 & \phi_3
    \end{pmatrix}   
    \end{bmatrix} \\
    & dxdy \\
    & = \frac{|K|}{3} \times \\
    & \begin{bmatrix}
      Uvec1(1)& 0 & 0 & Uvec1(2) & 0 & 0 \\
      0 & Uvec1(1) & 0 &  0 & Uvec1(2) & 0\\
      0 & 0 & Uvec1(1) &  0 & 0 &  Uvec1(2)\\
     Uvec2(1)& 0 & 0 & Uvec2(2) & 0 & 0 \\
      0 & Uvec2(1) & 0 &  0 & Uvec2(2) & 0\\
      0 & 0 & Uvec2(1) &  0 & 0 &  Uvec2(2)
    \end{bmatrix}
\end{align*}

生成NA矩阵的Matlab代码如下:
\begin{lstlisting}
%% 计算左端矩阵c(u^(l), u^(l), v):NA
NA1 = sparse(Nu,Nu);
NA2 = sparse(Nu,Nu);
NA3 = sparse(Nu,Nu);
NA4 = sparse(Nu,Nu);
sumUvec1 = zeros(NT,1,2);
sumUvec2 = zeros(NT,1,2);
test2 = sumUvec1(:,:,1);
for i = 1:NT
    sumUvec1(i,:)=reshape(u0vec(i,1:3),1,3)*reshape(Dlambda(i,:),3,2);
    sumUvec2(i,:)=reshape(u0vec(i,4:6),1,3)*reshape(Dlambda(i,:),3,2);
end
for i = 1:3
    for j = 1:3
        % 局部节点到全局节点
        ii = double(elem2dof(:,i));
        jj = double(elem2dof(:,j));
        if(i==j)
            NA1 = NA1 + sparse(ii,jj,sumUvec1(:,:,1).*area(:)/3,Nu,Nu); 
            NA2 = NA2 + sparse(ii,jj,sumUvec1(:,:,2).*area(:)/3,Nu,Nu); 
            NA3 = NA3 + sparse(ii,jj,sumUvec2(:,:,1).*area(:)/3,Nu,Nu); 
            NA4 = NA4 + sparse(ii,jj,sumUvec2(:,:,2).*area(:)/3,Nu,Nu); 
        end 
    end
end
NA = [NA1 NA2; NA3 NA4];
clear NA1 NA2 NA3 NA4
\end{lstlisting}


\subsection{推导$c(\mathbf{u}_h^{(l-1)}, \mathbf{u}_h^{(l)}, \mathbf{v}_h)$,矩阵NB}

\begin{align*}
    & c(\mathbf{u}_h^{0}, \mathbf{u}_h, \mathbf{v}_h)\\ 
    & = \iint_K(\vec{u} _h^0 \cdot \nabla) \vec{u}_h\cdot \vec{v}_hdxdy\\ 
    & = \iint_K(u_{h1}^0 \partial_x + u_{h2}^0 \partial_y) \vec{u}_h \cdot \vec{v}_h dxdy\\
    & = \iint_K \begin{pmatrix}
     u_{h1}^0\partial _xu_1 + u_{h2}^0\partial _yu_1 \\
     u_{h1}^0\partial _xu_2 + u_{h2}^0\partial _yu_2
    \end{pmatrix}
    \cdot \begin{pmatrix}
     v_1\\
    v_2
    \end{pmatrix}dxdy\\
    & = \iint_K \begin{pmatrix}
     \vec{u}_h^0 \cdot (\nabla u_1) \\
     \vec{u}_h^0 \cdot (\nabla u_2)
    \end{pmatrix}
    \cdot \begin{pmatrix}
     v_1\\
    v_2
    \end{pmatrix}dxdy\\
    & = \iint_K (\vec{u}_h^0 \cdot (\nabla u_1)) v_1 + (\vec{u}_h^0 \cdot (\nabla u_2)) v_2dxdy.
\end{align*}

因此
\begin{align*}
    & c(\boldsymbol{u}_h^{(l-1)}, \boldsymbol{u}_h^{(l)}, \boldsymbol{v}_h) \\
    & = \iint_K (\vec{u}_h^0 \cdot (\nabla u_1)) v_1 + (\vec{u}_h^0 \cdot (\nabla u_2)) v_2dxdy\\
    & = \iint_K \begin{pmatrix}
      \phi_1&  \phi_2 &  \phi_3 & 0 & 0 & 0\\
      0 & 0 & 0 & \phi_1 & \phi_2 & \phi_3
    \end{pmatrix}
    \begin{pmatrix}
     u_{11}^0\\
     u_{12}^0\\
     u_{13}^0\\
     u_{21}^0\\
     u_{21}^0\\
     u_{21}^0
    \end{pmatrix} \\
    & \cdot 
    \begin{pmatrix}
     \nabla \phi_1 & \nabla \phi_2 & \nabla \phi_3
    \end{pmatrix}
    \begin{pmatrix}
     u_{11}\\
     u_{12}\\
     u_{13}
    \end{pmatrix}
    \begin{pmatrix}
     v_{11} & v_{12} & v_{13}
    \end{pmatrix}
    \begin{pmatrix}
     \phi_1\\
     \phi_2\\
    \phi_3
    \end{pmatrix}\\
    & + 
    \begin{pmatrix}
      \phi_1&  \phi_2 &  \phi_3 & 0 & 0 & 0\\
      0 & 0 & 0 & \phi_1 & \phi_2 & \phi_3
    \end{pmatrix}
    \begin{pmatrix}
     u_{11}^0\\
     u_{12}^0\\
     u_{13}^0\\
     u_{21}^0\\
     u_{21}^0\\
     u_{21}^0
    \end{pmatrix} \\
    & \cdot 
    \begin{pmatrix}
     \nabla \phi_1 & \nabla \phi_2 & \nabla \phi_3
    \end{pmatrix}
    \begin{pmatrix}
     u_{21}\\
     u_{22}\\
     u_{23}
    \end{pmatrix}
    \begin{pmatrix}
     v_{21} & v_{22} & v_{23}
    \end{pmatrix}
    \begin{pmatrix}
     \phi_1\\
     \phi_2\\
    \phi_3
    \end{pmatrix}dxdy\\
    & = \begin{pmatrix}
      v_1^T& v_2^T
    \end{pmatrix}
    \begin{pmatrix}
      NB1& O_{3 \times 3}\\
      O_{3 \times 3} & NB1
    \end{pmatrix}
    \begin{pmatrix}
     \boldsymbol{u}_1\\
     \boldsymbol{u}_2
    \end{pmatrix}
\end{align*}
其中
\begin{align*}
    & NB1 \\
    & = \iint_K 
    \begin{pmatrix}
     \phi_1\\
     \phi_2\\
    \phi_3
    \end{pmatrix}
    \begin{pmatrix}
     u_{11}^0 \phi_1 + u_{12}^0 \phi_2  + u_{13}^0 \phi_3 
    & u_{21}^0 \phi_1 + u_{22}^0 \phi_2  + u_{23}^0 \phi_3 
    \end{pmatrix}dxdy\\
    & \begin{pmatrix}
     \nabla \phi_1 & \nabla \phi_1 & \nabla \phi_1
    \end{pmatrix} \\
    & = \frac{|K|}{3}\begin{pmatrix}
      u_{11}^0& u_{21}^0 \\
    u_{12}^0& u_{22}^0 \\
     u_{13}^0& u_{23}^0 
    \end{pmatrix} 
    \begin{pmatrix}
     \nabla \phi_1 & \nabla \phi_1 & \nabla \phi_1
    \end{pmatrix}\\
    &  = \frac{|K|}{3}\begin{pmatrix}
        u_{11}^0& u_{21}^0 \\
      u_{12}^0& u_{22}^0 \\
       u_{13}^0& u_{23}^0 
      \end{pmatrix} 
      \begin{pmatrix}
       \frac{\partial \phi_1}{\partial x} & \frac{\partial \phi_2}{\partial x}  & 
       \frac{\partial \phi_3}{\partial x}  \\
       \frac{\partial \phi_1}{\partial y} & \frac{\partial \phi_2}{\partial y}  & 
       \frac{\partial \phi_3}{\partial y} 
      \end{pmatrix}
\end{align*}
因此
$$
NB1_{ij} = \frac{|K|}{3}(u_{1i}^0 \frac{\partial \phi_j}{\partial x}+u_{2i}^0 \frac{\partial \phi_j}{\partial y}) 
$$
设
$$
newU0vec = 
\begin{pmatrix}
    u_{11}^0& u_{21}^0 \\
    u_{12}^0& u_{22}^0 \\
     u_{13}^0& u_{23}^0
\end{pmatrix}
$$

生成矩阵NA的Matlab代码如下:
\begin{lstlisting}
%% 计算左端矩阵c(u^(l-1), u^(l), v):NB
NB = sparse(Nu,Nu);
newU0vec = reshape(u0vec,NT,3,2);
for i = 1 : 3
    for j = 1 : 3
        % 局部节点到全局节点
        ii = double(elem2dof(:,i));
        jj = double(elem2dof(:,j));
        NBij = (newU0vec(:,i,1).*Dlambda(:,1,j) + ...
            newU0vec(:,i,2).*Dlambda(:,2,j)).*area(:)/3;
        NB = NB + sparse(ii,jj,NBij,Nu,Nu);
    end
end
clear NBij
NB = blkdiag(NB,NB);
\end{lstlisting}

\subsection{推导$c(\mathbf{u}_h^{(l-1)}, \mathbf{u}_h^{(l-1)}, \mathbf{v}_h)$,矩阵NF}
\begin{align*}
    & c(\mathbf{u}_h^{0}, \mathbf{u}_h^0, \mathbf{v}_h)\\ 
    & = \iint_K(\vec{u} _h^0 \cdot \nabla) \vec{u}_h^0\cdot \vec{v}_hdxdy\\ 
    & = \iint_K(u_{h1}^0 \partial_x + u_{h2}^0 \partial_y) \vec{u}_h^0 \cdot \vec{v}_h dxdy\\
    & = \iint_K \begin{pmatrix}
     u_{h1}^0\partial _xu_1^0 + u_{h2}^0\partial _yu_1^0 \\
     u_{h1}^0\partial _xu_2^0 + u_{h2}^0\partial _yu_2^0
    \end{pmatrix}
    \cdot \begin{pmatrix}
     v_1\\
    v_2
    \end{pmatrix}dxdy\\
    & = \iint_K \begin{pmatrix}
     \vec{u}_h^0 \cdot (\nabla u_1^0) \\
     \vec{u}_h^0 \cdot (\nabla u_2^0)
    \end{pmatrix}
    \cdot \begin{pmatrix}
     v_1\\
    v_2
    \end{pmatrix}dxdy\\
    & = \iint_K (\vec{u}_h^0 \cdot (\nabla u_1^0)) v_1 + (\vec{u}_h^0 \cdot (\nabla u_2^0)) v_2dxdy.
\end{align*}

因此

\begin{align*}
    & c(\boldsymbol{u}_h^{(l-1)}, \boldsymbol{u}_h^{(l-1)}, \boldsymbol{v}_h) \\
    & = \iint_K (\vec{u}_h^0 \cdot (\nabla u_1^0)) v_1 + (\vec{u}_h^0 \cdot (\nabla u_2^0)) v_2dxdy\\
    & = \iint_K \begin{pmatrix}
      \phi_1&  \phi_2 &  \phi_3 & 0 & 0 & 0\\
      0 & 0 & 0 & \phi_1 & \phi_2 & \phi_3
    \end{pmatrix}
    \begin{pmatrix}
     u_{11}^0\\
     u_{12}^0\\
     u_{13}^0\\
     u_{21}^0\\
     u_{21}^0\\
     u_{21}^0
    \end{pmatrix} \\
    & \cdot 
    \begin{pmatrix}
     \nabla \phi_1 & \nabla \phi_2 & \nabla \phi_3
    \end{pmatrix}
    \begin{pmatrix}
     u_{11}^0\\
     u_{12}^0\\
     u_{13}^0
    \end{pmatrix}
    \begin{pmatrix}
     v_{11} & v_{12} & v_{13}
    \end{pmatrix}
    \begin{pmatrix}
     \phi_1\\
     \phi_2\\
    \phi_3
    \end{pmatrix}\\
    & + 
    \begin{pmatrix}
      \phi_1&  \phi_2 &  \phi_3 & 0 & 0 & 0\\
      0 & 0 & 0 & \phi_1 & \phi_2 & \phi_3
    \end{pmatrix}
    \begin{pmatrix}
     u_{11}^0\\
     u_{12}^0\\
     u_{13}^0\\
     u_{21}^0\\
     u_{21}^0\\
     u_{21}^0
    \end{pmatrix} \\
    & \cdot 
    \begin{pmatrix}
     \nabla \phi_1 & \nabla \phi_2 & \nabla \phi_3
    \end{pmatrix}
    \begin{pmatrix}
     u_{21}^0\\
     u_{22}^0\\
     u_{23}^0
    \end{pmatrix}
    \begin{pmatrix}
     v_{21} & v_{22} & v_{23}
    \end{pmatrix}
    \begin{pmatrix}
     \phi_1\\
     \phi_2\\
    \phi_3
    \end{pmatrix}dxdy\\
    & = \begin{pmatrix}
      v_1^T& v_2^T
    \end{pmatrix}
    \begin{pmatrix}
      NF1_{3 \times 1}\\
      NF2_{3 \times 1}
    \end{pmatrix}
    \begin{pmatrix}
     \boldsymbol{u}_1\\
     \boldsymbol{u}_2
    \end{pmatrix}
\end{align*}
其中
\begin{align*}
    & NF1\\
 & =\iint_K
 \begin{pmatrix}
  \phi_1\\
  \phi_2\\
 \phi_3
 \end{pmatrix} 
 \begin{pmatrix}
   u_{11}^0\phi_1 + u_{12}^0\phi_2 + u_{13}^0\phi_3&
 u_{21}^0\phi_1 + u_{22}^0\phi_2 + u_{23}^0\phi_3
 \end{pmatrix}dxdy
 [sumUvec1]^T
 \\
 & = \frac{|K|}{3}
 \begin{pmatrix}
   u_{11}^0 &  u_{21}^0 \\
    u_{12}^0 &  u_{22}^0 \\
    u_{13}^0 &  u_{23}^0
 \end{pmatrix}[sumUvec1]^T\\
 & = \frac{|K|}{3}\mathbf{newU0vec}  [sumUvec1]^T
 \end{align*}

因此
$$
NF= \frac{|K|}{3}\begin{pmatrix}
    \mathbf{newU0vec} [sumUvec1]^T\\
    \mathbf{newU0vec} [sumUvec2]^T
\end{pmatrix}
$$

生成矩阵NF的Matlab代码如下:
\begin{lstlisting}
%% 计算右端矩阵c(u^(l-1), u^(l-1), v):NF
nf1 = zeros(NT,3,1);
nf2 = zeros(NT,3,1);
for i = 1 : NT
    nf1(i,:,:) = reshape(newU0vec(i,:,:),3,2)*reshape(sumUvec1(i,:,:),2,1)...
    .*area(i)/3;
    nf2(i,:,:) = reshape(newU0vec(i,:,:),3,2)*reshape(sumUvec2(i,:,:),2,1)...
    .*area(i)/3;
end
NF1 = accumarray(elem2dof(:),[nf1(:,1); nf1(:,2); nf1(:,3)],[Nu 1]);
NF2 = accumarray(elem2dof(:),[nf2(:,1); nf2(:,2); nf2(:,3)],[Nu 1]);
NF = [NF1;NF2];
\end{lstlisting}
\end{document}