\documentclass{ctexart}
% \usepackage[UTF-8]{ctex}
\usepackage{amsmath}
\usepackage{booktabs}
\usepackage{multirow}
\usepackage{tabularx}

\usepackage{booktabs,multirow,longtable}

\usepackage{listings}
\usepackage{color}

% 插入Matlab代码设置
% \usepackage[framed, numbered, autolinebreaks, useliterate]{mcode}

%导言区插入下面三行
\usepackage{graphicx} %插入图片的宏包
\usepackage{float} %设置图片浮动位置的宏包
\usepackage{subfigure} %插入多图时用子图显示的宏包

% % 参考文献
% % 注意参考文献请用Biber编译,不能使用BiberTex或者BiberTex-8
\usepackage[backend=biber,style=gb7714-2015]{biblatex}
\addbibresource[location=local]{bib/sample.bib}

%% 引用网页
\usepackage{url}


% \definecolor{dkgreen}{rgb}{0,0.6,0}
% \definecolor{gray}{rgb}{0.5,0.5,0.5}
% \definecolor{mauve}{rgb}{0.58,0,0.82}

% \lstset{frame=tb,
%   language=Matlab,
%   aboveskip=3mm,
%   belowskip=3mm,
%   showstringspaces=false,
%   columns=flexible,
%   basicstyle={\small\ttfamily},
%   numbers=left,
%   numberstyle=\tiny\color{gray},
%   keywordstyle=\color{blue},
%   commentstyle=\color{dkgreen},
%   stringstyle=\color{mauve},
%   breaklines=true,
%   breakatwhitespace=true,
%   tabsize=3
% }



\title{CR-P0方法求解二维Navier-Stokes方程}
\author{谢文进}
\date{\today}
\begin{document}
\maketitle
\section{理论推导}
\subsection{问题描述}
对于Navier-Stokes问题

\begin{equation}
    \label{NavierStokes}
    \left\{\begin{matrix}
        \partial_t\mathbf{u} + (\mathbf{u} \cdot \nabla)\mathbf{u}    - 
\nu \Delta \mathbf{u}  + \nabla p = f, \quad \text{in } \Omega\\ 
        \nabla \cdot \mathbf{u} = 0 = h, \quad \text{in } \Omega \\
        \mathbf{u} = g = 0, \quad \text{on } \partial \Omega
\end{matrix}\right.
\end{equation}
其中,取$\nu = 1$。这里函数$h(\mathbf{x},t)$为预留函数接口,方便后续处理随机项。

第一步,先求解如下方程
\begin{equation}
    \label{NavierStokes01}
    \left\{\begin{matrix}
        (\mathbf{u} \cdot \nabla)\mathbf{u}    - 
 \nu \Delta \mathbf{u}  + \nabla p = f, \quad \text{in } \Omega\\ 
         \nabla \cdot \mathbf{u} = 0 = h, \quad \text{in } \Omega \\
         \mathbf{u} = g = 0, \quad \text{on } \partial \Omega
 \end{matrix}\right.
\end{equation}
其中
$$
\nabla = \begin{pmatrix}
    \partial _x\\
   \partial _y
   \end{pmatrix},
$$

$$
(\mathbf{u} \cdot \nabla)\mathbf{u} = \begin{pmatrix}
    u_1 \partial _x + u_2 \partial _y
    \end{pmatrix}
    \begin{pmatrix}
     u_1\\
    u_2
    \end{pmatrix}
    =\begin{pmatrix}
     u_1\partial _xu_1+u_2\partial _yu_1\\
     u_1\partial _xu_2+u_2\partial _yu_2
    \end{pmatrix}
$$

对于此非线性问题,需要使用牛顿迭代方法,即找到$u^{(l)} \in H^1(\Omega) \times H^1(\Omega)$,
$p^{(l)} \in L^2(\Omega)$使得
\begin{equation}
    \left\{\begin{matrix}
        c(\mathbf{u}^{(l)}, \mathbf{u}^{(l-1)}, \mathbf{v} ) 
+ c(\mathbf{u}^{(l-1)}, \mathbf{u}^{(l)}, \mathbf{v} ) +
a(\mathbf{u}^{(l)} , \mathbf{v}) 
+  b(\mathbf{v},p^{(l)} )\\
= (f,\mathbf{v}) + c(\mathbf{u}^{(l-1)}, \mathbf{u}^{(l-1)}, \mathbf{v} ),\\ 
        b(\mathbf{u}^{(l)}, q ) = 0 = (h, q) .
\end{matrix}\right.
\end{equation}
其中
$$
a(\mathbf{u,v})=\nu(\nabla \mathbf{u},\nabla \mathbf{v}), 
\quad b(\mathbf{v},p)=-(\text{div } \mathbf{v},p)
$$
和用CRP0方法求解Stokes方程中的算子一样,而
$$
c(\mathbf{w,u,v})= ((\mathbf{w} \cdot \nabla)\mathbf{u}, \mathbf{v}).
$$

\subsection{推导$c(\mathbf{u}_h^{(l-1)}, \mathbf{u}_h^{(l)}, \mathbf{v}_h)$}
为了方便描述,这里假设牛顿迭代前一步得到的为$\mathbf{u}_h^0$,现在要求的是$\mathbf{u}_h$,则
\begin{align*}
    & c(\mathbf{u}_h^{0}, \mathbf{u}_h, \mathbf{v}_h)\\ 
    & = \iint_K(\vec{u} _h^0 \cdot \nabla) \vec{u}_h\cdot \vec{v}_hdxdy\\ 
    & = \iint_K(u_{h1}^0 \partial_x + u_{h2}^0 \partial_y) \vec{u}_h \cdot \vec{v}_h dxdy\\
    & = \iint_K \begin{pmatrix}
     u_{h1}^0\partial _xu_1 + u_{h2}^0\partial _yu_1 \\
     u_{h1}^0\partial _xu_2 + u_{h2}^0\partial _yu_2
    \end{pmatrix}
    \cdot \begin{pmatrix}
     v_1\\
    v_2
    \end{pmatrix}dxdy\\
    & = \iint_K \begin{pmatrix}
     \vec{u}_h^0 \cdot (\nabla u_1) \\
     \vec{u}_h^0 \cdot (\nabla u_2)
    \end{pmatrix}
    \cdot \begin{pmatrix}
     v_1\\
    v_2
    \end{pmatrix}dxdy\\
    & = \iint_K (\vec{u}_h^0 \cdot (\nabla u_1)) v_1 + (\vec{u}_h^0 \cdot (\nabla u_2)) v_2dxdy.
\end{align*}
其中
$$
\vec{u}_h = \begin{pmatrix}
    \phi_1&  \phi_2& \phi_3 & 0 & 0 & 0\\
    0 & 0 & 0 & \phi_1 & \phi_2 & \phi_3
  \end{pmatrix} 
  \begin{pmatrix}
   u_{11}^0\\
   u_{12}^0\\
   u_{13}^0\\
   u_{21}^0\\
   u_{22}^0\\
   u_{23}^0
  \end{pmatrix},
$$

$$
\nabla u_i = \begin{pmatrix}
    \nabla \phi_1& \nabla \phi_2 & \nabla \phi_3
  \end{pmatrix}_{2 \times 3}
  \begin{pmatrix}
   u_{i1}\\
   u_{i2}\\
  u_{i3}
  \end{pmatrix},i=1,2
$$

$$
v_i = 
\begin{pmatrix}
  v_{i1}& v_{i2}  & v_{i3}
\end{pmatrix}
\begin{pmatrix}
 \phi _1\\
  \phi _2\\
 \phi _3
\end{pmatrix}
\quad i=1,2.
$$

\end{document}